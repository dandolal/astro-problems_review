\documentclass[12pt]{article}

\usepackage[utf8]{inputenc}
\usepackage[russian]{babel}
\usepackage[OT1]{fontenc}

\usepackage{amsmath}
\usepackage{amsfonts}

\usepackage[
	a4paper,
	left 	=	1	cm,
	right 	=	1	cm,
	top		=	2	cm,
	bottom 	= 	2	cm
]{geometry}

%%%%%%%%%%%%%%%%%%%%%%%%%%%% Колонтитулы %%%%%%%%%%%%%%%%%%%%%%%%%%%%
\usepackage{fancyhdr}
\pagestyle{fancy}
\fancyfoot[L]{\copyright\,Efrem}
\fancyfoot[R]{\copyright\,Астрономический кружок им.~Е.\,П.~Левитана}
\fancyhead[R]{\thepage}
\fancyhead[L]{<<Антипланеты>>}
\fancyfoot[C]{}

\usepackage{graphicx}
\usepackage{wasysym}

\setlength{\parindent}{0pt}


\begin{document}
	\begin{footnotesize}
		\textbf{Источник:}~УТС.Зима, Астрофизический дивертисмент\\
		\textbf{Авторство:}~Шепелев А.С.\\
		\textbf{Тематика:}~Небесная механика\\
		\textbf{Год:}~2017\\
		\textbf{Класс:}~9\,--11\\
		\textbf{Рейтинг:}~117
	\end{footnotesize}
	
	\vspace{1pc}
	{\footnotesize \textbf{Условие:}}
	
	Астрономы Лупа и Пупа живут на~антипланетах, обращающихся вокруг звезды
    с~массой  $M_{\star} \simeq 10 M_\odot$ по~эллиптической орбите
    с~фокальным параметром~$p = 0.3~а.\,е.$
    и~эксцентриситетом~$e = 0.72$.
    Как и~полагается антипланетам, время от~времени центральная звезда
    находится точно между ними. В этот момент~$X$
    истинная аномалия~$\nu$ планеты Пупы составляет $237^\circ$.

    Однажды кто-то опять всё перепутал, и~в~момент~$X$~центральная звезда
    бесследно исчезла, при этом модули скоростей планет уменьшились
    в~$217$~раз. Установите, с~каким периодом~$T$ планеты
    бедных астрономов будут обращаться в~отсутствие звезды. 
    Известно, что планеты относятся к~классу горячих Юпитеров
    с массой $M \simeq M_{\jupiter}$.
	
	\vspace{1pc}
	{\footnotesize \textbf{Решение:}}
	
	Пусть индекс $p$ будет обозначать величины, относящиеся к~Пупе и~его планете, а $l$~--- относящиеся, соответственно, к Лупе.

    Длина радиус-вектора для~каждой из~планет
    в~момент исчезновения звезды определяется из уравнения эллипса в полярных координатах. Учитывая, что они~находились
    на~одной прямой, получаем
    \[
        r_{p1} = \frac{p}{1 + e \cos \nu} = \frac{0.3}{1 + 0.72 \cos 237^\circ} = 0.494~а.\,е.;
    \]
    \[
        r_{l1} = \frac{p}{1 + e \cos \left( \nu - 180^\circ \right)} = \frac{0.3}{1 + 0.72 \cos \left( 237^\circ - 180^\circ \right)}
            = 0.215~а.\,е.
    \]
    Как известно, для большой полуоси орбиты $a$ справедливо следующее выражение:
    \[
    	a = \dfrac{p}{1-e^2} = 0.623~а.\,е.
    \]
    Из~теоремы косинусов для~треугольников $\triangle F_1 F_2 P$
    и~$\triangle F_1 F_2 L$, определяются расстояния от~планет Пупы и Лупы соответственно до~второго фокуса:
    \begin{align}
    	r_{p2} &= \sqrt{4 a^2 e^2 + r_{p1}^2 -
    	       2 \cdot 2ae \cdot r_{p1}
    	       \cos \left( \nu - 180^\circ \right)} = 0.752 \, \text{а.}\,\text{е.},\\
    	r_{l2} &= \sqrt{4 a^2 e^2 + r_{l1}^2 -
    	       2 \cdot 2ae \cdot r_{l1}
    	       \cos \left( 360^\circ - \nu \right)} = 1.030 \, \text{а.}\,\text{е.}.
    \end{align}

    Угол между направлениями на~фокусы орбиты
    для~Пупы и~Лупы определяется из~теоремы синусов для~тех~же треугольников:
    \begin{align}
        \alpha_p &= \arcsin\left(\frac{2 a e
            \sin \left( \nu - 180^\circ \right)}
            {r_{p2}}\right) = 89.6^\circ; \\
        \alpha_l &= \arcsin\left(\frac{2 a e
            \sin \left( 360^\circ - \nu \right)}{r_{l2}}\right) = 46.9^\circ.
    \end{align}
    Из~<<бильярдного>> (оптического) свойства эллипса получаем угол 
    между радиус-векторами планет и векторами их скоростей:
    \begin{align}
    	\beta_p &= \frac{180^\circ+ \alpha_p}{2} = \frac{180^\circ + 89.6^\circ}{2} = 134.8^\circ; \\
    	\beta_l &= \frac{180^\circ- \alpha_l}{2} = \frac{180^\circ - 46.9^\circ}{2} = 66.6^\circ. 
    \end{align} 
    Интеграл энергии поможет найти модули скоростей планет
    после события~$X$: 
    \begin{align}
    	V_P &= \frac{1}{217} \sqrt{GM_\star
            \left( \frac{2}{r_{p1}} - \frac{1}{a}\right)} = 0.68~км/с;
    \\
        V_L &= \frac{1}{217} \sqrt{GM_\star
            \left( \frac{2}{r_{l1}} - \frac{1}{a}\right)} = 1.20~км/с.
    \end{align}

    Далее находим относительную тангенциальную и~радиальную скорости
    планет:
    \begin{align}
    	V_\tau &= V_p \sin \beta_p + V_l \sin \beta_l = 0.68 \sin 134.8^\circ + 1.20 \sin 66.6^\circ = 1.59~км/с; \\
    	V_r &= V_p \cos \beta_p + V_l \cos \beta_l = 0.68 \cos 134.8^\circ + 1.20 \cos 66.6^\circ = 0.00~км/с.
    \end{align}
    Здесь уже учтена \emph{ненулевая скорость центра масс} системы.

    Мысленно перенеся всю массу в~одну из~планет, <<фиксируем>> её
    и~пускаем вторую обращаться вокруг неё
    с~расчётной относительной скоростью $V_\text{отн}$.
    \[
        V_\text{отн}^2 = V_\tau^2 + V_r^2 = (1.59\cdot 10^3)^2 = 2.53 \cdot 10^6.
    \]
    Из интеграла энергии получаем большую полуось $a'$ образовавшейся системы:
    \[
    	a' = \left( \frac{2}{r_{p1} + r_{l1}} -
            \frac{V_\text{отн}^2}{2GM} \right)^{-1}
             = 0.355 \, \text{a.}\,\text{е.}.
    \]
     
    Осталось найти период из~обобщённого третьего закона Кеплера: 
    \[
        T = T_\oplus \sqrt{\frac{a'^3}{a_\oplus^3} 
            \frac{M_\odot}{2M}} \approx 4.7 \text{ года}.
    \]
	
\end{document}